\chapter{Evaluation}
\label{cha:evaluation}

Even though this project was mainly thought of as a development thesis,
the potential for interesting conclusions that 
could be drawn from this experiment was so high that it was decided 
to do some research with the resulting eLearning tool as well.

This chapter of my thesis presents the whole evaluation of the 
developed program. It also serves as proof of my exposed theory in 
the abstract section, that technology can be easily adopted by 
a general public to assist and facilitate learning habits.

The chapter is organized as follow: I'll start by describing 
the methodology used to evaluate the skill, explaining how 
it profiles the users and their behaviors. I will follow 
the aforementioned by the statistical data provided 
by the experiment, showing the scope of the users' 
characteristics and how they represent certain classes. 
I'll then finish the chapter with the valuable results 
and suggestions received with feedback in addition to 
my academical interpretation of the accomplished.


\section{Methodology}

The methodology behind the survey is heavily inspired by the TAM \ref{section:tam}. 
The features it tries to outline, 
are the user's perceived usefulness and ease-of-use. 
In addition to these attributes, my survey has a heavy focal 
point on defining the user's profile, or establishing a 
classification class or target group which shares the stipulated 
characteristics of the aforementioned profile. The latter's 
objective is to draw supplementary conclusions, such as to 
discover whether or not a certain group is more prone to a 
wide adoption of the/a new technology and is ready to promptly 
change its learning customs.
Furthermore, the survey inquires its user about suggestions 
on improvements or other features that could make the 
software more appealing to the target class and a more 
global scope of attainable users.
For logistical reasons, mostly the incapacitating restrictions 
originated by the COVID-19 virus, 
this survey and experiment was only performed on 
a sorted out set of six individuals, 
all representing different classification classes, namely different 
age groups, professional backgrounds and studying purposes 
(academic, professional or recreational).
As such, the extent of the survey its 
much larger than the commonly used TAM studies, 
for it contains much more extensive and 
particular questions, in order to derive prevailing 
and accurate deductions.


\section{Survey Candidates Sampling}
\label{section:sampling}
For logistical reasons, mostly the incapacitating restrictions 
originated by the COVID-19 virus, 
this survey and experiment was only performed on 
a sorted out set of six individuals, 
all representing different classification classes, namely different 
age groups, professional backgrounds and studying purposes 
(academic, professional or recreational).
As such, the extent of the survey its 
much larger than the commonly used TAM studies, 
for it contains much more extensive and 
particular questions, in order to derive prevailing 
and accurate deductions.

The most important distinctions, which represent fundamental 
differences between users
were found in:

\begin{itemize}
    \item \textbf{Age: } Perhaps the most important from all the 
	parameters categorized
        in the sampling process, age has an almost direct influence 
	on all the other criteria. If the test subject is in its 
	twenties, then he will more likely 
        be using the software to practice for school or some 
	certification needed. The subject will also more likely be 
	more knowledgeable of newer technologies. If
        the subject is over his seventies, he will very 
	unlikely be learning for
        professional purposes, etc. 
	This class will be divided in five main age groups that share 
	learning purposes and habits: 6-18 as a pupil, who will most likely
	be studying either for school or a certification. 19-28 as a student
	or junior starting his career mostly gathering certification, academical
	degrees or learning for some hobby interest. 29-50 as a professional
	that either learns for his professional career or out of interest.
	51-65 as a senior worker that will either need to learn some new skill
	for his work or for a hobby. 66+ as a retired individual who will 
	most likely only learn and practice something out of interest with
	absolutely no professional goals in mind.

    \item \textbf{Purpose/Goal: } This was indeed a very important 
	distinction in the sampling process. This parameter 
	was divided in three categories.
	Professional/career, academical/school/certification or 
	hobby/recreational. Even though the attitude towards learning
	can be the same or different across these different categories,
	for example, a student might be equally inspired learning 
	for an university topic as he is for something he might be
	learning for entertainment purposes only, the type of
	motivation in the background is very different. 
	Someone studying for professional reasons, being motivated or
	not, is to expect similar consequences regarding their work.
	Either to get a promotion, or to improve the workflow, but the type
	of expectations, clearly affect the \textit{drive} that guides the corresponding learning 
	process.
	

    \item \textbf{Professional/Academic Background: }
	Also an important mean of measurement is the academical background 
	of the interviewee. If the user has a extensive educational context,
	he is expected to have gathered a framework of methods to assist in
	his learning habits, while someone without it will more likely resort
	to different tools.
	This field could be separated with four main categories, namely:
	high school or equivalency diplomas (pupils would be automatically
	included in this field), professional degree or academic degree, where
	the latter will create distinctions for minor/major and PHD.

    \item \textbf{Practicing Habits: }
	Although this point is heavily influenced by the others, as are all 
	the five by each other, the habits a learner has to enhance his 
	knowledge of the given topic is very important regarding his approach
	to A: sensibility to change and B: perspective on use of technology.
	This class will be split into people that study less than 1 hour a week,
	from 1 to 3 hours a week or more than 3 hours a week.

    \item \textbf{Attitude Towards Technology: }
	Perhaps the most important factor, although already mentioned and 
	deliberately influenced by the practicing habits, the attitude towards
	technology plays an essential role in the users apprehension of the
	system and potential change of behaviour.  
	This field will be dissected into three groups, someone with negative 
	view on technology, that for instance actively tries to avoid usage 
	of technology as much as possible, someone with neutral view, that ends
	up using technology very widely, but is not particularly kin on it, and
	someone that has a very positive view, a person that will be a typically
	early adopter of new technologies.

\end{itemize}


A major field used to categorize and distinguish users in a survey is gender.
I made a conscience choice to avoid this differentiation, as although it might
regularly show interesting dissimilarities, a sample of 6 test subjects is much
too small, and any differences will likely be biased by the other parameters 
mentioned above, which in my opinion play a much bigger role in the learning 
habits and practices of learners.


As one might observe, the combination of
all different evaluation parameters with their given subclasses means at least
540 test subjects would need to be part of the survey, which would be impossible given the
conditions mentioned in the beginning of the chapter. For that reason, the users
chosen to represent the classes are the 6 most distant points in the 5 
dimensional space of these characteristics.

As such, the first selected 6 test subjects were: 

\subsection{Toddler}
The first test subject was a nine year old toddler, who is learning English to
improve his grades at school. Is profile can be shown in the following table:

\begin{center}
	\begin{tabular}{ | c | c | }
	\hline
		Age & 6-18 \\
	\hline
		Goal & Extracurricular Consolidation \\
	\hline
		Educational Background & High School \\
	\hline
		Practicing Habits & <1h a week \\
	\hline
		Attitude Towards Technology & Positive \\
	\hline

	\end{tabular}
\end{center}

\subsection{University Student}
The second test subject was a twenty two year old university student practicing
for a bachelor's subject. The following table shows his profile:

\begin{center}
	\begin{tabular}{ | c | c | }
	\hline
		Age & 19-28 \\
	\hline
		Goal & Academical \\
	\hline
		Educational Background & High School \\
	\hline
		Practicing Habits & 1-3h a week \\
	\hline
		Attitude Towards Technology & Neutral \\
	\hline

	\end{tabular}
\end{center}


\subsection{Junior Getting Certification}
The third test subject was a twenty six year old young musician with a masters 
degree studying to get her drivers license. The table shows her profile:

\begin{center}
	\begin{tabular}{ | c | c | }
	\hline
		Age & 29-50 \\
	\hline
		Goal & Certification \\
	\hline
		Educational Background & Minor/Major Degree \\
	\hline
		Practicing Habits & <1h a week \\
	\hline
		Attitude Towards Technology & Negative \\
	\hline

	\end{tabular}
\end{center}

\subsection{Professional Learning For Hobby}
The fourth test subject was a thirty seven year old professional who wants to
study music theory to enhance her piano playing skills. Her profile is shown
below:

\begin{center}
	\begin{tabular}{ | c | c | }
	\hline
		Age & 29-50 \\
	\hline
		Goal & Hobby \\
	\hline
		Educational Background & PHD \\
	\hline
		Practicing Habits & 1-3h a week \\
	\hline
		Attitude Towards Technology & Neutral \\
	\hline

	\end{tabular}
\end{center}


\subsection{Senior Office Worker}
The fifth test subject was a sixty two year old office worker who wants 
to practice knowledge needed at work, as the leader of the library department
of the Portuguese Ministry of Education, 
and was opened to try a new 
methodology, even though she had 
a negative view on technology. The table shows other parameters:

\begin{center}
	\begin{tabular}{ | c | c | }
	\hline
		Age & 51-65 \\
	\hline
		Goal & Professional \\
	\hline
		Educational Background & Professional Degree \\
	\hline
		Practicing Habits & 3h+ a week \\
	\hline
		Attitude Towards Technology & Negative \\
	\hline

	\end{tabular}
\end{center}

\subsection{Pensioner}
The last test subject was a seventy two year old pensioner who is very kin on
the use of technology and an early adopter and used it to practice his 
studying of the Chinese culture. The profile is shown below:


\begin{center}
	\begin{tabular}{ | c | c | }
	\hline
		Age & 66+ \\
	\hline
		Goal & Hobby \\
	\hline
		Educational Background & Minor/Major Degree \\
	\hline
		Practicing Habits & 3h+ a week \\
	\hline
		Attitude Towards Technology & Positive \\
	\hline

	\end{tabular}
\end{center}


\section{Interview Process}
The interview process was planed to follow the following steps:

\begin{enumerate}
    \item Meet with the interviewee either personally or via Skype call.
    \item Introduce the concept behind the experiment and explain the procedures.
    \item Start with the pre-questionnaire. 
    \item Give the URL of the web application and setup the alexa skill. If the interview
        takes place personally, no setup is required, if not, the user's email needs to be
        added to the collaborators lista in the alexa developer console.
    \item Allow the interviewee to explore the website and skill as he pleases only intervening 
        if the interfaces are not clear or help is needed.
    \item Observe the interaction and take notes, also limiting the amount of time spent in the
        web application (15 minutes) and alexa skill (10 minutes).
    \item Finish off the experiment by filling the post-questionnaire.
\end{enumerate}

 
\section{Survey}

As mentioned above the survey was adapted from generalized 
TAM questions, which are used on experiments with 100+ users, to a more detailed
set of questions which focus mainly on three big aspects. The learner's 
attitude towards technology, the ease-of-use and the perceived usefulness of
the developed system.

There is a short profile section that inquires the user about the list of 
parameters mentioned earlier in this chapter, followed by a pre-questionnaire
to be inquired before the use of the application 
and a post-questionnaire. Besides asking more extensive questions about the
overall experience, the survey also urges to learner for suggestions that could
improve the efficiency of the still young and incomplete software.

The following can be observed here \ref{appendix:listing1}.


The results are going to be divided into each one of the participants, since 
they yield very different conclusions.



\subsection{Toddler}
The six year old toddler that used the smart speaker application to improve
his English as a foreign language skills, showed very positive results. He
already had a very positive view on technologies and had a tablet as mentioned
by his primary teacher to assist on some homework. 

The toddler had a positive experience and was thrilled to use such a fun piece
of technology. From his feedback not many conclusions can be drawn, for the simple
fact that he still lacks experience and knowledge of other methodologies and
represents a highly volatile class of learners.

\subsection{University Student}
The university student was very pleased with her experience with the smart speaker
and felt it could prove useful in many daily life situations, though discarding it as a 
main means of practicing and studying for the future.

She also proposed very interesting features that could be added to the web application
to improve the overall experience. It was suggested that the web platform should implement
more social features. Each user should be able to have a profile where her topics would be exposed to 
others and one should also be able to follow and and see which user created and uploaded which topics,
textbooks and questions. The suggestion very strongly hinted that a community of practice should
be built and the web application should take advantage of the useful social networking capability.


\subsection{Junior Getting Certification}
This participant had some trouble setting up and getting started with the alexa skill, although she 
thought the web application was of easy and practical use.

The junior had many suggestions on how to improve the user experience and 
the effectiveness of the tool which focused mainly on the smart speaker. The user suggested that
it didn't make much sense to divide content from web and alexa applications and that one should be
able to upload the desired content immediately via voice to the alexa, simplifying the whole procedure.
This suggestion opened many questions about the whole architecture of the tool, but the 
proposal to add it as an extra feature to the skill sounded very interesting.

\subsection{Professional Learning For Hobby}
This professional had a good experience. It took some time to understand the whole mechanism, but  afterwards it worked smoothly. One important suggestion was made, in which the learner proposed that
users should be able to rate questions, either by difficulty or by quality. Such as to sort them 
according to difficulty when using the skill and to be able to delete some other questions that would
be considered poorly formulated by the community sharing them. This, she suggested, should only be done
on the web application, since the alexa interface would become otherwise to complex.


\subsection{Senior Office Worker}
The senior office worker, who already had a very negative opinion on newer technologies had difficulty
setting up the environment and could solely proceed with my accompaniment. By the end of the experiment
she admitted it could be useful but wasn't convinced she would keep practicing with it. 

A suggestion was made, so that textbooks could be used in the web application more widely, by either
marking passages, creating questions just for some paragraphs, and have the alexa skill read them out load
before prompting the user with the questions.


\subsection{Pensioner}
And finally the pensioner had a very positive experience. He needed some rounds to understand the engine
of the alexa skill but was very happy and excited to keep studying his topics of interest using it.
This result wasn't very surprising, for he already had one smart speaker in the first place and was 
looking for an excuse to use it in a regular basis.



\section{Summary}
In summary in this chapter the whole evaluation procedure and model were presented. Firstly the 
technology acceptance model's variation, then the selection of learner individuals that could represent
a broader class, who would have similar outcomes, followed by the description of the survey and the
experiment's outcome. The following chapter will present my inference and interpretation of the results
and all proceedings observed.