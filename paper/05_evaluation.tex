\chapter{Evaluation}
\label{cha:evaluation}

This chapter of my thesis presents the whole evaluation of the developed tool. It also serves as proof of my exposed theory in the abstract section, that technology can be easily adopted by a general public to assist and facilitate learning habits.

The chapter is organized as follow: I'll start by describing the methodology used to evaluate the skill, explaining how it profiles the users and their behaviours. I will follow the aforementioned by the statistical data provided by the experiment, showing the scope of the users' characteristics and how they represent certain classes. I'll then finish the chapter with the valuable results and suggestions received with feedback in addition to my academical interpretation of the accomplished.


\section{Methodology}

The methodology behind the survey is heavily inspired by the TAM (Technology Acceptance Model). 
The features it tries to outline, are the user's perceived usefulness and ease-of-use. 
In addition to these attributes, my survey has a heavy focal point on defining the user's 
profile, or establishing a classification class or target group which shares the stipulated 
characteristics of the aforementioned profile. The latter's objective is to draw 
supplementary conclusions, such as to discover whether or not a certain group is more prone 
to a wide adoption of the/a new technology and is ready to promptly change its learning customs.
Furthermore, the survey inquires its user about suggestions on improvements or other features 
that could make the software more appealing to the target class and a more global scope of 
attainable users. For logistical reasons, mostly the incapacitating restrictions originated 
by the COVID-19 virus, this survey and experiment was only performed on a sorted out set 
of 5 individuals, all representing different classification classes, namely different age 
groups, professional backgrounds and studying purposes (academic, professional or recreational).
As such, the extent of the survey its much larger than the commonly used TAM studies, 
for it contains much more extensive and particular questions, in order to derive prevailing 
and accurate deductions.


\section{Survey Candidates Sampling}

For logistical reasons, mainly the impacts of COVID-19 in university infrastructures,
the survey and experiment was performed on a much smaller subset of users, than that 
expected by the regular Technology Acceptance Model surveys. Instead of testing the 
software with 100+ users, it was only optional to narrow down and assembly a total of
6 test subjects. For that reason, a comprise was found and before sampling the subjects
a graph was created to represent the main classes and targets of the potential audience 
of this application.

SHOW IMAGE HERE.

The most important distinctions, which represent fundamental differences between users
were found in 

\begin{itemize}
    \item \textbf{Age: } Perhaps the most important from all the parameters categorized
        in the sampling process, age as an almost direct influence on all the other 
        criteria. If the test subject is in its twenties, then he will more likely 
        be using the software to practice for school or some certification needed. The
        subject will also more likely be more knowledgeable of newer technologies. If
        the subject in over its seventies, it will very unlikely be working for
        professional purposes, etc. 

    \item \textbf{Purpose/Goal: } This is a very important distinction in the sampling 
        process. This parameter was divided in three categories.

    \item \textbf{Professional/Academic Background: }

    \item \textbf{Practicing Habits: }

    \item \textbf{Attitude Towards Technology: }

\end{itemize}

As such, the first selected 6 test subjects were: 

 
\section{Survey Results}
The application was tested by a number of people to give their feedback about it. The audience were provided with the................... Once that was done, they were asked to complete a survey questionnaire (see Appendix 1: Survey Questions). 

The following sections present the results of the survey.


\subsection{Participants general information}

\subsection{System Usefulness Measurement}


\subsection{System Usability Scale}
This section contains questions regarding System Usability Scale. These questions try to assess whether the implemented system is easy to use or not and the survey results are listed below:

\begin{itemize}
\item I think that I would like to use this system frequently.
\item I thought the system was easy to use and understand.
\item I would imagine that one would learn to use this system very quickly.
\item I felt confident using the system.
\end{itemize}

Response to above listed questions show............


\section{Validity/Efficiency Measurement}

During the evaluation..........


\section{Summary}
In this chapter, ......................
