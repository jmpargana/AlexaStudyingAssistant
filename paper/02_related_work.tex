\chapter{Related Work}
\label{cha:relatedwork}

This section covers all the related work of what influences this thesis.
The most essential concepts are Communities of Practice, Informal Learning, 
Technology Acceptance Model 
and Question-Based Distributed Architectures.
These topics will be exposed and explored throughout this chapter, 
reviewing the sources I found most valuable and my opinions and 
readings of each theme.


\section{Communities of Practice}
\section{Informal Learning}
\section{Question-Based Distributed Architectures}
\section{Technology Acceptance Model}

\section{Tacit Knowledge}

\section{Micro-Learning}

Micro-Learning -> in absence of curriculum based approaches 
(structured and organized)
=> in opposition to valuable elements like didactic, organizational and social 
aspects **REFERENCE** (a web service architecture for social micro-learning)

Definition: brevity, small learning units or short-term learning activities
**REFERENCE**
1. is self-contained, self-explanatory and can be presented without further context,
2. comprises a single learning activity that can be performed within seconds, and
3. provides immediate performance feedback.

Baumgartner proposes a model of a micro-learner which focuses on informal learning
and learners themselves, a student needs to
1. absorb basic knowledge about a topic or subject (Learning I) - Behaviourism
2. actively acquire knowledge in a self-determined matter (Learning II) - Cognitivism
3. finally be able to construct knowledge (Learning III) - constructivism
Typically, micro-learning systems focus on the Learning I phase.

\subsection{Behaviorism}

Behaviorism is a systematic approach, methodology and theory of psychology which
was developed in the beginning of the twentieth century. Behaviorism treats 
animals and Humans alike and attempts to explain how they function 
purely by observation, thus completely disregarding any internal components such 
as the mind with thinking and emotions. The main goal of Behaviorism is to predict
and control behavior via observation. Opposed to other theories of psychology, the
behaviorists believe their subjects are born as a "tabula rasa" or with a 
clean state and learn by reinforcement or punishment. According to this theory,
humans or animals have responses to stimuli based purely on their conditioning
and surrounding environment. This conditioning is divided mostly into two 
categories, the classic and operand or behavior conditioning, which explains or 
treats their reactions in different ways. The classic conditioning explains the 
response to stimulus by combining the individual's history, motivation while 
the operand controls and manipulates the result by reinforcing it with a 
consequence such as punishment or reward.

Behaviorism tries to link the subjects' responses to the entire surrounding 
environment, without creating any assumptions of the internal organisms. In 
learning, this can be considered as the most basic form of practicing, where 
reinforcement and repetition shape the future behavior. In the context of 
Micro-Learning or Micro-Content this means that the first phase, in which the 
learner gathers samples he neglects all the thinking and emotions from the learning
process and sticks to his instincts.



\subsection{Cognitivism}

The psychology theory of Cognitivism emerged in the fifties mainly as a reaction 
to both challenge and dispute the then settled Behaviorism.

-Attention (together with memory are essential) information 
selection and processing -> (mental and perceptual attention) importance to 
achieve goals
-how information is organized and contextualized into the students existing schema
-"" is retrieved upon recall
-learning is a process depending on what the learner already knows the his method
of acquiring new knowledge.
- knowledge acquisition is the activity of internal codification of mental 
structures (planning, goal setting, and organizational strategies Shell, 1980)
-memory is essential -> information is stored and organized in a meaningful manner.
-information can be organized in a different way through analogies or other 
hierarchical relationships to facilitate memory acquisition
-once information is stored and implemented in different contexts and conditions
it is said to be transferred. (not stored in memory but used)



\subsection{Constructivism}

There are three main types of constructivism, cognitive by Jean Piaget, social
by Lev Vygotsky and radical.
Cognitive -> is actively constructed by learners based on their existing cognitive
structures, therefore learning is relative to their stage of cognitive development.


\begin{enumerate}
\item Knowledge is constructed, rather than innate, or passively absorbed 
    learners build new knowledge upon the foundation of previous learning.
\item learning is an active process
    learners only construct meaning though active engagement with the world with
    experiments or real-world problem solving.
    Information may be passively received, but understanding cannot be.
    It must come from making meaningful connections between prior and new knowledge.
\item All knowledge is socially constructed
    matter of sharing and negotiating constituted knowledge.
\item All knowledge is personal
\item Learning exists in the mind and does not match any real world reality.

\end{enumerate}



\section{Summary}

In this chapter, I discussed various state of the art techniques used.....................
