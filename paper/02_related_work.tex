\chapter{Related Work}
\label{cha:relatedwork}

This section covers all the related work of what influences this thesis.
The most essential concepts are Communities of Practice, Informal Learning, 
Technology Acceptance Model 
and Question-Based Distributed Architectures.
These topics will be exposed and explored throughout this chapter, 
reviewing the sources I found most valuable and my opinions and 
readings of each theme.


\section{Communities of Practice}
\section{Informal Learning}
\section{Question-Based Distributed Architectures}
\section{Technology Acceptance Model}

\section{Tacit Knowledge}

\section{Micro-Learning}

Micro-Learning -> in absence of curriculum based approaches 
(structured and organized)
=> in opposition to valuable elements like didactic, organizational and social 
aspects **REFERENCE** (a web service architecture for social micro-learning)

Definition: brevity, small learning units or short-term learning activities
**REFERENCE**
1. is self-contained, self-explanatory and can be presented without further context,
2. comprises a single learning activity that can be performed within seconds, and
3. provides immediate performance feedback.

Baumgartner proposes a model of a micro-learner which focuses on informal learning
and learners themselves, a student needs to
1. absorb basic knowledge about a topic or subject (Learning I) - Behaviourism
2. actively acquire knowledge in a self-determined matter (Learning II) - Cognitivism
3. finally be able to construct knowledge (Learning III) - constructivism
Typically, micro-learning systems focus on the Learning I phase.

\subsection{Behaviourism}



\subsection{Cognitivism}

\subsection{Constructivism}

There are three main types of constructivism, cognitive by Jean Piaget, social
by Lev Vygotsky and radical.
Cognitive -> is actively constructed by learners based on their existing cognitive
structures, therefore learning is relative to their stage of cognitive development.

\begin{enumerate}
\item Knowledge is constructed, rather than innate, or passively absorbed 
    learners build new knowledge upon the foundation of previous learning.
\item learning is an active process
    learners only construct meaning though active engagement with the world with
    experiments or real-world problem solving.
    Information may be passively received, but understanding cannot be.
    It must come from making meaningful connections between prior and new knowledge.
\item All knowledge is socially constructed
    matter of sharing and negotiating constituted knowledge.
\item All knowledge is personal
\item Learning exists in the mind and does not match any real world reality.

\end{enumerate}



\section{Summary}

In this chapter, I discussed various state of the art techniques used.....................
