\chapter{Related Work}
\label{cha:relatedwork}

This section covers all the related work of what influences this thesis.
The most essential concepts are Communities of Practice, Informal Learning, 
Technology Acceptance Model 
and Question-Based Distributed Architectures.
These topics will be exposed and explored throughout this chapter, 
reviewing the sources I found most valuable and my opinions and 
readings of each theme.


\section{Communities of Practice}
\section{Informal Learning}
\section{Question-Based Distributed Architectures}
\section{Technology Acceptance Model}

\section{Tacit Knowledge}

\section{Micro-Learning}

Micro-Learning -> in absence of curriculum based approaches 
(structured and organized)
=> in opposition to valuable elements like didactic, organizational and social 
aspects **REFERENCE** (a web service architecture for social micro-learning)

Definition: brevity, small learning units or short-term learning activities
**REFERENCE**
1. is self-contained, self-explanatory and can be presented without further context,
2. comprises a single learning activity that can be performed within seconds, and
3. provides immediate performance feedback.

Baumgartner proposes a model of a micro-learner which focuses on informal learning
and learners themselves, a student needs to
1. absorb basic knowledge about a topic or subject (Learning I) - Behaviourism
2. actively acquire knowledge in a self-determined matter (Learning II) - Cognitivism
3. finally be able to construct knowledge (Learning III) - constructivism
Typically, micro-learning systems focus on the Learning I phase.

\subsection{Behaviorism}

Behaviorism is a systematic approach, methodology and theory of psychology which
was developed in the beginning of the twentieth century. Behaviorism treats 
animals and Humans alike and attempts to explain how they function 
purely by observation, thus completely disregarding any internal components such 
as the mind with thinking and emotions. The main goal of Behaviorism is to predict
and control behavior via observation. Opposed to other theories of psychology, the
behaviorists believe their subjects are born as a "tabula rasa" or with a 
clean state and learn by reinforcement or punishment. According to this theory,
humans or animals have responses to stimuli based purely on their conditioning
and surrounding environment. This conditioning is divided mostly into two 
categories, the classic and operand or behavior conditioning, which explains or 
treats their reactions in different ways. The classic conditioning explains the 
response to stimulus by combining the individual's history, motivation while 
the operand controls and manipulates the result by reinforcing it with a 
consequence such as punishment or reward.

Behaviorism tries to link the subjects' responses to the entire surrounding 
environment, without creating any assumptions of the internal organisms. In 
learning, this can be considered as the most basic form of practicing, where 
reinforcement and repetition shape the future behavior. In the context of 
Micro-Learning or Micro-Content this means that the first phase, in which the 
learner gathers samples he neglects all the thinking and emotions from the learning
process and sticks to his instincts.



\subsection{Cognitivism}

The psychology theory of Cognitivism emerged in the late fifties mainly as a 
reaction to both challenge and dispute the then settled Behaviorism.
Cognitivism as a learning philosophy opposes Behaviorism in that instead of 
guiding the learners towards the desired direction, it uses feedback to assist 
learners to create accurate mental connections of the desired learning Schemas.

Cognitivism denies the "black box" approach of behaviorist and seeks to understand
the Human mind during the learning process. It endeavors to recognize the mental 
patterns needed to store new information and generate individual logic.
According to cognitivists, students have existing structures or Schemas, and 
the information absorbed during the learning process gets selected, processed
and organized to fit and enhance these memory patterns. \cite{johnson}
Attention and memory are essential elements of the learning process, for they 
function as connection to the already existing Schemas and help map and contextualize
new information. 

This process of internal codification of mental structures, following the steps
of planning, goal setting and organization strategies \cite{johnson}, are the 
responsible for knowledge acquisition and conservation. Learning is though a 
process that depends highly on the what the learner already knows and his or her
methods of acquiring new knowledge. Information can be organized in different 
ways according to the already existing structures, and it is the teachers 
job to convey data in a way that facilitates the conversion in to this structures, 
such as by analogies or other hierarchical relationships. Once information is 
stored and implemented in different contexts it is said to be transferred. \cite{schunk}

Cognitivism justifies and represents a second level of learning which is retained
in long-term memory and used every time the stored information is applied.




\subsection{Constructivism}

Constructivism os am evem further interpretation theory of the learning process 
as explained by Behaviorism and Congnitivism. According to Constructivism, leaners
build or construct their own knowledge through personal experience. They give meaning
to information and assemble it upon the foundation of previous learned knowledge, 
hence the name Constructivism.

According to this theory, new knowledge gets absorbed influenced solely on the 
existing one. The learner constructs or stacks the knowledge understanding the new 
information and concepts molded by what he has learned previously. This process is
active \cite{adler_1970} as all types of learning processes are, where aided or 
unaided the leaner must constantly interpret and analyze the new information,
relating it to the existing one, only then creating a new layer of understanding 
when enlightened. This process of active engagement is individual, thus the same 
lessons can create very different outcomes, since the learner decides how to construct
new meaning combined the new information with the already available mental models.

Teachers who wish to convey new meaning as Constructivists, tend to use active 
engaging teaching methods, such as experiments or real-world problems to incite
learners to utilize their understanding and layer them with the new information.

According to John Dewey \cite{dewey_1938}, this process is a highly social one. 
Interaction between learners will enhance their construction of meaning, the same
way physical work will be assembled more effectively if collaborating.
John Dewey defends also that even unaided forms of learning, via books or other 
resources require indirect socially interaction in order for the learning process 
to work effectively. This idea is also shared by a major contributer to this theory,
Lev Vygotsky \cite{vygotsky_1978}, who developed the Social Constructivism, where 
he justifies with a community, the process of making meaning, also labeled by
Cognitivists as meaning transfer, can occur more effectively with sharing and 
negotiating.

Learning exists only in the mind and is an iterative process that keeps updating the
mental models to adapt to new information. Learning is the constant construction 
of our own interpretation of reality.



\section{Summary}

In this chapter, I discussed various state of the art techniques used.....................
